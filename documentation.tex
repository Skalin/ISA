\documentclass[11pt,a4paper]{report}
\usepackage[left=2cm,text={17cm, 24cm},top=3cm]{geometry}
\usepackage[utf8]{inputenc}
\usepackage{blindtext}
\usepackage[czech]{babel}
\usepackage[T1]{fontenc}
\usepackage{times}
\usepackage{hyphenat}
\usepackage{hyperref}
\usepackage{amsmath}
\usepackage{graphicx}
\graphicspath{ {images/} }
\usepackage{float}
\usepackage{indentfirst}
\setlength\parindent{1cm}
\usepackage[noline,linesnumbered,titlenumbered,ruled,czech]{algorithm2e}
\usepackage{multirow}
\usepackage{enumitem}
\usepackage[bottom]{footmisc}
\usepackage{lscape}
\usepackage{epsfig}
\usepackage{url}
\hypersetup{linkcolor=magenta}
\renewcommand\thesection{\arabic{section}}

\newcommand\myuv[1]{\quotedblbase #1\textquotedblleft}
\def\myauthor{Dominik Skála, xskala11@stud.fit.vutbr.cz}
\def\myleader{Vedoucí projektu:}
\def\myleadername{Ing. Martin Holkovič, iholkovic@fit.vutbr.cz}
\def\mytitle{Síťové aplikace a správa sítí \\ Dokumentace projektu}

\begin{document}
\thispagestyle{empty}
\begin{center}
\Huge
\textsc{Fakulta informačních technologií Vysoké učení technické v~Brně}

\vspace{\stretch{0.382}}
    \begin{center}
        \includegraphics[scale=0.2]{./images/vut/logo.png}
    \end{center}
\mytitle
\\
\vspace{\stretch{0.218}}
\LARGE POP3 server
\vspace{\stretch{0.4}}
\end{center}

{\Large
\begin{flushleft}
\myleader \hfill
\myleadername
\end{flushleft}
 \begin{flushright}
 \today \hfill
\myauthor
\end{flushright}}

\clearpage

\tableofcontents % prehled sekci

\clearpage
\section{Zadání}
Vytvořte komunikující aplikaci podle konkrétní vybrané specifikace pomocí síťové knihovny BSD sockets (pokud není ve variantě zadání uvedeno jinak). Projekt bude vypracován v jazyce C/C++. Pokud individuální zadání nespecifikuje vlastní referenční systém, musí být projekt přeložitelný a spustitelný na serveru merlin.fit.vutbr.cz.
\subsection{Varianta POP3 server}
Vytvorte program popser, ktorý bude plniť úlohu POP3 [1] serveru (ďalej iba server). Na server sa budú pripájať klienti, ktorý si pomocou protokolu POP3 sťahujú dáta zo serveru. Server bude pracovať s emailmi uloženými vo formáte IMF [4] v adresárovej štruktúre typu Maildir [2].

Vašim cieľom je teda vytvoriť program, ktorý bude vedieť pracovať s emilami uloženými v adresárovej štruktúre Maildir. Program bude pracovať iba s jednou poštovou schránkou. Takto uložené emaily budú prostredníctvom protokolu POP3 poskytované jedinému používateľovi.
\section{Uvedení do problematiky}
\subsection{POP3 server}
\subsubsection{Serverové stavy}
\subsubsection{E-mailové operace}
\subsubsection{Formy autorizace}
\section{Řešení projektu}
\subsection{Jazyk}
Pro implementaci byl zadán jazyk C/C++. V projektu byl vybrán primárně jazyk C++, zejména pro jeho vylepšenou práci s paměti a prací s řetězci. Místy jsou však i využity operace a standardy, které by byly používány primárně v jazyce C.
\subsection{Vývojové prostředí}
Pro implementaci byl využit verzovací nástroj git, zejména tedy jeho populární provozovatel Github s využitím studentské licence.\par
Pro vývoj v jazyce C/C++ byl využit nástroj CLion od firmy JetBrains s využitím studentské licence.
\subsection{Popis řešení}
\subsubsection{Přeložení a spuštění}
POP3 server, dále jen server, je možné přeložit za pomocí makefile příkazem: make nebo make all. Příkaze make provede pouze znovupřeložení serveru, make all provede vyčištění adresáře od starých verzí serveru, překlad serveru a následné smazání objektových souborů.
V případě potřeby je možné využít i příkazu: make clean-obj a make clean. Příkaz make clean provede smazání starých verzí serveru, make clean-obj provede smazání starých objektových souborů.\par
Server se následně spouští příkazem: ./popser [-h] [-a PATH] [-c] [-p PORT] [-d PATH] [-r]\par
Server, má tři formy běhu:
\begin{enumerate}
\item ./popser -h -- pouhý výpis nápovědy
\item ./popser -r -- pouhé provedení resetu serveru do stavu před prvním spuštěním serveru
\item ./popser -a PATH -p PORT -d PATH [-c] [-r] -- běžný režim běhu serveru
\end{enumerate}
\subsubsection{Popis argumentů při spuštění}
Každý argument při spuštění má svůj význam a je nutné jej validovat. \par

\begin{description}
\item [-h] je argument spouštějící nápovědu serveru. Je možné jej zadat s jakýmkoliv jiným argumentem, pokud je však zadán, dojde pouze k výpisu nápovědy a ukončení serveru. Tento argument je možné validně využít ve všech formách běhu serveru.
V serveru je toto řešeno jako přepínač, který jakmile je aktivován, dojde k vytištění nápovědy a ukončení serveru s návratovým kódem \textbf{EXIT\_SUCCESS} (\ref{itm:exitsuccess}).
\item [-c] je argument, který umožňuje výběr typu autorizace v serveru. Pokud není zadán, je možné využít pouze autorizace přes příkazy APOP, pokud byl zadán, je možné využít pouze autorizace přes kombinaci příkazů USER/PASS. Tento argument je možné validně využít pouze v třetím režimu běhu.
V serveru je toto řešeno jako přepínač, který jakmile je aktivován, rozhoduje povoluje jednu z těchto autorizačních metod.
\item [-r] je argument provádějící reset serveru, tedy návrat do stavu před prvním spuštěním serveru, tj. dojde k načtení všech e-mailů ze speciálního konfiguračního souboru, v serveru je vždy pojmenován: mail.cfg a je umístěn v root složce serveru. Po provedení resetu se e-maily přesunou ze složky cur do new, změní se název na původní a server se tváří jako by nebyl spuštěn. Toto neplatí pro již smazané e-maily, ty nelze obnovit. Pokud je tento argument předán samostatně, dojde k resetu serveru a ukončení serveru s návratovou hodnotou \textbf{EXIT\_SUCCESS} (\ref{itm:exitsuccess}), pokud není zadán samostatně a je v kombinaci s třetí variantou spuštění, přejde se k dalšímu spouštění serveru. Tento argument je možné validně využít pouze v druhém a třetím režimu běhu.
\item [-a PATH] je argumentem určujícím cestu k souboru s autorizačními údaji uživatele serveru. Tento argument je povinný v třetím režimu běhu. Pokud není v tomto režimu běhu zadán, dojde k ukončení běhu serveru s návratovým kódem \textbf{EXIT\_FAILURE} (\ref{itm:exitfailure}), pokud je zadán, dojde k načtení a validaci souboru s autorizačními údaji.
\item [-d PATH] je argumentem určujícím cestu k e-mailovému adresáři. Tento argument je povinný v třetím režimu běhu. Pokud není v tomto režimu běhu zadán, dojde k ukončení běhu serveru s neúspěšným návratovým kódem \textbf{EXIT\_FAILURE} (\ref{itm:exitfailure}), pokud je zadán, je cesta uložena, ověřena validita cesty a až v pozdější fázi serveru jsou testovány bližší podmínky.
\item [-p PORT] je argumentem určujícím číslo portu, na kterém server poběží. Tento argument je povinný v třetím režimu běhu. Pokud není v tomto režimu běhu zadán, dojde k ukončení běhu serveru s návratovým kódem \textbf{EXIT\_FAILURE} (\ref{itm:exitfailure}), pokud je zadán, číslo portu je uloženo a je s ním dále pracováno až při spouštění serveru.
\end{description}
\subsubsection{Běh serveru}
Po spuštění serveru dojde podle režimu běhu k různému chování. \par
Po spuštění serveru v režimu běhu 1 dojde k vytištění nápovědy a ukončení serveru s návratovým kódem \textbf{EXIT\_SUCCESS} (\ref{itm:exitsuccess}).
Po spuštění serveru v režimu běhu 2 dojde k resetu serveru a k ukončení programu s návratovým kódem \textbf{EXIT\_SUCCESS} (\ref{itm:exitsuccess}), pokud nedošlo k žádným problémům a následně se smaže soubor mail.cfg.\par
Po spuštění serveru v režimu běhu 3 dojde k validaci všech argumentů, následně dojde k validaci autorizačního souboru. Poté dojde k vytvoření socketu, jeho nastavení, k nabindování serverového socketu. Poté server přejde do stavu naslouchání na daném socketu. Pokud vše proběhlo v pořádku, server je spuštěn a čeká na připojení od klienta.\par
Po připojení klienta se klient musí autorizovat, je tedy ve stavu autorizace. Dokud neproběhne úspěšná autorizace, nedostane klient přístup k e-mailům. Jakmile se autorizuje, je mu vyhrazen přístup k e-mailové složce nad kterou může výhradně on provádět e-mailové operace. V tomto momentě je v transakčním stavu. Po odpojení klienta dojde k přesunu do stavu aktualizace, ze kterého úspěšném dokončení přejde do stavu čekání na dalšího klienta. Server je schopen přijímat zároveň více klientů, výhradní přístup je však možné dát v jeden moment pouze jednomu klientovi.
\subsection{Implementace řešení}
\subsubsection{Návratové stavy}
V serveru jsou využity dva návratové stavy. \textbf{EXIT\_SUCCESS} a \textbf{EXIT\_FAILURE}. Každý z těchto stavů je makrem pro číselnou hodnotu, bylo vhodně využito těchto maker pro transformovanost a přeložitelnost na různých systémech. Různé typy systémů mohou považovat různé návratové hodnoty za jiný stav. Využitím těchto maker tomuto lze jednoduše předejít.
\begin{description}
\item [EXIT\_SUCCESS] je brán jako hodnota úspěšného ukončení běhu programu. Standardně se na Unixových systémech jedná o číselnou hodnotu \textbf{0}.
\label{itm:exitsuccess}
\item [EXIT\_FAILURE] je brán jako hodnota neúspěšného ukončení běhu programu. Standardně se na Unixových systémech jedná o číselnou hodnotu \textbf{1}.
\label{itm:exitfailure}
\end{description}
\subsubsection{Ukončení běhu serveru}
\section{Základní informace o serveru}
\section{Návod k použití}
\bibliographystyle{plain}
\bibliography{references}
\end{document}
