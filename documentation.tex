\documentclass[11pt,a4paper]{report}
\usepackage[left=2cm,text={17cm, 24cm},top=3cm]{geometry}
\usepackage[utf8]{inputenc}
\usepackage{blindtext}
\usepackage[czech]{babel}
\usepackage[T1]{fontenc}
\usepackage{times}
\usepackage{hyphenat}
\usepackage{hyperref}
\usepackage{amsmath}
\usepackage{graphicx}
\graphicspath{ {images/} }
\usepackage{float}
\usepackage{indentfirst}
\setlength\parindent{1cm}
\usepackage[noline,linesnumbered,titlenumbered,ruled,czech]{algorithm2e}
\usepackage{multirow}
\usepackage{enumitem}
\usepackage[bottom]{footmisc}
\usepackage{lscape}
\usepackage{listings}
\usepackage{epsfig}
\usepackage{url}
\hypersetup{linkcolor=magenta}
\renewcommand\thesection{\arabic{section}}
\setcounter{tocdepth}{4}
\setcounter{secnumdepth}{4}

\newcommand\myuv[1]{\quotedblbase #1\textquotedblleft}
\def\myauthor{Dominik Skála, xskala11@stud.fit.vutbr.cz}
\def\myleader{Vedoucí projektu:}
\def\myleadername{Ing. Martin Holkovič, iholkovic@fit.vutbr.cz}
\def\mytitle{Síťové aplikace a správa sítí \\ Dokumentace projektu}

\begin{document}
    \thispagestyle{empty}
    \begin{center}
        \Huge
        \textsc{Fakulta informačních technologií Vysoké učení technické v~Brně}

        \vspace{\stretch{0.382}}
        \begin{center}
            \includegraphics[scale=0.2]{./images/vut/logo.png}
        \end{center}
        \mytitle
        \\
        \vspace{\stretch{0.218}}
        \LARGE POP3 server
        \vspace{\stretch{0.4}}
    \end{center}

    {\Large
    \begin{flushleft}
        \myleader \hfill
        \myleadername
    \end{flushleft}
    \begin{flushright}
        \today \hfill
        \myauthor
    \end{flushright}}

    \clearpage

    \tableofcontents % prehled sekci

    \clearpage
    \section{Zadání}
    Vytvořte komunikující aplikaci podle konkrétní vybrané specifikace pomocí síťové knihovny BSD sockets (pokud není ve variantě zadání uvedeno jinak). Projekt bude vypracován v jazyce C/C++. Pokud individuální zadání nespecifikuje vlastní referenční systém, musí být projekt přeložitelný a spustitelný na serveru merlin.fit.vutbr.cz.
    \subsection{Varianta POP3 server}
    Vytvorte program popser, ktorý bude plniť úlohu POP3 [1] serveru (ďalej iba server). Na server sa budú pripájať klienti, ktorý si pomocou protokolu POP3 sťahujú dáta zo serveru. Server bude pracovať s emailmi uloženými vo formáte IMF [4] v adresárovej štruktúre typu Maildir [2].

    Vašim cieľom je teda vytvoriť program, ktorý bude vedieť pracovať s emilami uloženými v adresárovej štruktúre Maildir. Program bude pracovať iba s jednou poštovou schránkou. Takto uložené emaily budú prostredníctvom protokolu POP3 poskytované jedinému používateľovi.

    \clearpage
    \section{Uvedení do problematiky}
    \subsection{POP3 server}
    \label{subsec:pop3server}
    \subsubsection{Serverové stavy}
    Server dosahuje 3 (respektive 4) stavů, mezi kterými postupně iteruje. Server dosahuje stavu: Čekání na připojení, Autorizace, Transakce, Aktualizace.
    \begin{center}
        \includegraphics[scale=0.8]{./images/popfsm.png}
    \end{center}
    \begin{description}
        \item [Čekání na připojení]
        \item [Autorizace]
        \item [Transakce]
        \item [Aktualizace]
    \end{description}
    \label{subsec:serverstatus}
    \subsubsection{Serverové odpovědi}
    \label{subsec:commresponse}
    \subsubsection{E-mailové operace}
    \label{subsec:mailoperations}
    \subsubsection{Formy autorizace}
    \label{subsec:authmethods}
    Server umožňuje dvě formy autorizace, užitím příkazu APOP nebo kombinací příkazů USER/PASS.
    \begin{description}
        \item [APOP] Během této formy autorizace je užita nonce (?? nonce), na základě které klient zašle serveru zašifrované heslo ve formátu: nonce.pass ("." je symbolem konkatenace řetězců). Pro šifrování je využito md5 algoritmu. Celý příkaz který klient zasílá je ve formátu: APOP name nonce.pass, kde name je uživatelské jméno a pass je heslo z autorizačního souboru (\ref{subsec:authconfig}).
        \item [USER/PASS] Během této formy autorizace je uživatelské jméno a heslo přenášeno v čisté formě serveru, klient po připojení k serveru nejprve pošle příkaz: USER name a pokud server klienta pozná, požaduje heslo, které klient dodá příkazem: PASS pass, kde name je uživatelské jméno a pass je heslo z autorizačního souboru (\ref{subsec:authconfig}).
    \end{description}

    \clearpage
    \section{Řešení projektu}
    \subsection{Jazyk}
    Pro implementaci byl zadán jazyk C/C++. V projektu byl vybrán primárně jazyk C++, zejména pro jeho vylepšenou práci s paměti a prací s řetězci. Místy jsou však i využity operace a standardy, které by byly používány primárně v jazyce C.
    \subsection{Vývojové prostředí}
    Pro implementaci byl využit verzovací nástroj git, zejména tedy jeho populární provozovatel Github s využitím studentské licence.\par
    Pro vývoj v jazyce C/C++ byl využit nástroj CLion od firmy JetBrains s využitím studentské licence.
    \subsection{Popis řešení}
    \subsubsection{Přeložení a spuštění}
    \label{subsec:makeandrun}
    POP3 server, dále jen server, je možné přeložit za pomocí makefile příkazem: make nebo make all. Příkaze make provede pouze znovupřeložení serveru, make all provede vyčištění adresáře od starých verzí serveru, překlad serveru a následné smazání objektových souborů.
    V případě potřeby je možné využít i příkazu: make clean-obj a make clean. Příkaz make clean provede smazání starých verzí serveru, make clean-obj provede smazání starých objektových souborů.\par
    Server se následně spouští příkazem: ./popser [-h] [-a PATH] [-c] [-p PORT] [-d PATH] [-r]\par
    Server, má tři formy běhu:
    \begin{enumerate}
        \item ./popser -h -- pouhý výpis nápovědy
        \item ./popser -r -- pouhé provedení resetu serveru do stavu před prvním spuštěním serveru
        \item ./popser -a PATH -p PORT -d PATH [-c] [-r] -- běžný režim běhu serveru
    \end{enumerate}
    \subsubsection{Popis argumentů při spuštění}
    Každý argument při spuštění má svůj význam a je nutné jej validovat. \par

    \begin{description}
        \item [-h] \label{itm:help} je argument spouštějící nápovědu serveru. Je možné jej zadat s jakýmkoliv jiným argumentem, pokud je však zadán, dojde pouze k výpisu nápovědy a ukončení serveru. Tento argument je možné validně využít ve všech formách běhu serveru.
        V serveru je toto řešeno jako přepínač, který jakmile je aktivován, dojde k vytištění nápovědy a ukončení serveru s návratovým kódem \textbf{EXIT\_SUCCESS} (\ref{itm:exitsuccess}).
        \item [-c] \label{itm:authorisation}je argument, který umožňuje výběr typu autorizace v serveru. Pokud není zadán, je možné využít pouze autorizace přes příkazy APOP, pokud byl zadán, je možné využít pouze autorizace přes kombinaci příkazů USER/PASS. Tento argument je možné validně využít pouze v třetím režimu běhu.
        V serveru je toto řešeno jako přepínač, který jakmile je aktivován, rozhoduje povoluje jednu z těchto autorizačních metod.
        \item [-r] \label{itm:reset}je argument provádějící reset serveru, tedy návrat do stavu před prvním spuštěním serveru, tj. dojde k načtení všech e-mailů ze speciálního konfiguračního souboru, v serveru je vždy pojmenován: mail.cfg a je umístěn v root složce serveru. Po provedení resetu se e-maily přesunou ze složky cur do new, změní se název na původní a server se tváří jako by nebyl spuštěn. Toto neplatí pro již smazané e-maily, ty nelze obnovit. Pokud je tento argument předán samostatně, dojde k resetu serveru a ukončení serveru s návratovou hodnotou \textbf{EXIT\_SUCCESS} (\ref{itm:exitsuccess}), pokud není zadán samostatně a je v kombinaci s třetí variantou spuštění, přejde se k dalšímu spouštění serveru. Tento argument je možné validně využít pouze v druhém a třetím režimu běhu.
        \item [-a PATH] je argumentem určujícím cestu k souboru s autorizačními údaji uživatele serveru. Tento argument je povinný v třetím režimu běhu. Pokud není v tomto režimu běhu zadán, dojde k ukončení běhu serveru s návratovým kódem \textbf{EXIT\_FAILURE} (\ref{itm:exitfailure}), pokud je zadán, dojde k načtení a validaci souboru s autorizačními údaji.
        \item [-d PATH] je argumentem určujícím cestu k e-mailovému adresáři. Tento argument je povinný v třetím režimu běhu. Pokud není v tomto režimu běhu zadán, dojde k ukončení běhu serveru s neúspěšným návratovým kódem \textbf{EXIT\_FAILURE} (\ref{itm:exitfailure}), pokud je zadán, je cesta uložena, ověřena validita cesty a až v pozdější fázi serveru jsou testovány bližší podmínky.
        \item [-p PORT] je argumentem určujícím číslo portu, na kterém server poběží. Tento argument je povinný v třetím režimu běhu. Pokud není v tomto režimu běhu zadán, dojde k ukončení běhu serveru s návratovým kódem \textbf{EXIT\_FAILURE} (\ref{itm:exitfailure}), pokud je zadán, číslo portu je uloženo a je s ním dále pracováno až při spouštění serveru.
    \end{description}
    \subsubsection{Běh serveru}
    Po spuštění serveru dojde podle režimu běhu k různému chování.\\ \par
    Po spuštění serveru v prvním režimu běhu dojde k vytištění nápovědy a ukončení serveru s návratovým kódem \textbf{EXIT\_SUCCESS} (\ref{itm:exitsuccess}). \par
    Po spuštění serveru v druhém režimu běhu dojde k resetu serveru a k ukončení programu s návratovým kódem \textbf{EXIT\_SUCCESS} (\ref{itm:exitsuccess}), pokud nedošlo k žádným problémům a následně se smaže soubor mail.cfg.\par
    Po spuštění serveru v třetím režimu běhu dojde k validaci všech argumentů, následně dojde k validaci autorizačního souboru. Poté dojde k vytvoření socketu, jeho nastavení, k nabindování serverového socketu. Poté server přejde do stavu naslouchání na daném socketu. Pokud vše proběhlo v pořádku, server je spuštěn a čeká na připojení od klienta. Blíže je toto specifikováno v sekci \ref{subsec:serverimplementation}.\par
    Po připojení klienta se klient musí autorizovat, je tedy ve stavu autorizace. Dokud neproběhne úspěšná autorizace, nedostane klient přístup k e-mailům. Jakmile se autorizuje, je mu vyhrazen přístup k e-mailové složce nad kterou může výhradně on provádět e-mailové operace. V tomto momentě je v transakčním stavu. Po odpojení klienta dojde k přesunu do stavu aktualizace, ze kterého úspěšném dokončení přejde do stavu aktualizace, ve kterém smaže e-maily pro smazání a odhlásí klienta a povolí přístup dalšímu klientovi ke složce maildiru. Následně se přesune do stavu čekání na dalšího klienta. Server je schopen přijímat zároveň více klientů, výhradní přístup je však možné dát v jeden moment pouze jednomu klientovi. Podrobně je běh serveru a jeho implementace řešena v sekci \ref{subsec:pop3implementation}.

    \clearpage
    \subsection{Implementace řešení}
    \subsubsection{Návratové stavy}
    V serveru jsou využity dva návratové stavy. \textbf{EXIT\_SUCCESS} a \textbf{EXIT\_FAILURE}. Každý z těchto stavů je makrem pro číselnou hodnotu, bylo vhodně využito těchto maker pro transformovanost a přeložitelnost na různých systémech. Různé typy systémů mohou považovat různé návratové hodnoty za jiný stav. Využitím těchto maker tomuto lze jednoduše předejít.
    \begin{description}
        \item [EXIT\_SUCCESS] je brán jako hodnota úspěšného ukončení běhu programu. Standardně se na Unixových systémech jedná o číselnou hodnotu \textbf{0}.
        \label{itm:exitsuccess}
        \item [EXIT\_FAILURE] je brán jako hodnota neúspěšného ukončení běhu programu. Standardně se na Unixových systémech jedná o číselnou hodnotu \textbf{1}.
        \label{itm:exitfailure}
    \end{description}
    \subsubsection{Soubor s autorizačními údaji}
    \label{subsec:authconfig}
    Soubor s autorizačními údaji obsahuje přístupové údaje uživatele k serveru, obsahuje vždy pouze uživatelské jméno a heslo v následujícím formátu:
    \begin{lstlisting}[frame=trBL]
        username = NAME
        pass = PASS
    \end{lstlisting}\par
    Položka NAME je jméno uživatele, položka PASS je přístupové heslo. Obě tyto položky jsou uloženy v nehashované formě.
    \subsubsection{Konfigurační soubor}
    \label{subsec:config}
    Konfigurační soubor e-mailů obsahuje veškeré e-maily, které po ukončení serveru (\ref{subsec:sigint}) server uloží do souboru mail.cfg, který je vždy uložen ve stejné podsložce jako mailserver samotný, kde každému e-mailu jsou vytvořeny dva řádky v jednoduchém formátu:
    \lstset{frameround=fttt}
    \begin{lstlisting}[frame=trBL]
        dir = PATH
        name = NAME
    \end{lstlisting}\par
    Položka PATH je neúplná cesta k souboru, kterou má server uloženu v paměti ve speciální struktuře mailStruct (\ref{subsec:mailstruct}). Tyto e-maily jsou vždy uloženy ve složce e-mailového adresáře, v podsložce /cur, vždy je tedy uložena pouze cesta bez /cur. \par
    Položka NAME je jméno e-mailu, které je taky uloženo v paměti ve speciální struktuře mailStruct (\ref{subsec:mailstruct}).\par
    Tento soubor je po zapnutí server načten vždy, jsou z něj vždy načteny e-maily. Pokud dojde k resetu (\ref{subsec:reset}), provede se přesun e-mailů do daných souborů do původních složek, tedy do složky e-mailového adresáře, do podsložky /new
    \subsubsection{Struktury a jejich kolekce}
    \label{subsec:threadstruct}
    Server využívá struktur dvou typů.\par
    Jednou z nich je struktura typu threadStruct, která je strukturou vlákna, obsahující potřebné informace pro komunikaci s daným klientem v daném vlákně. Tato struktura je definována následovně:
    \lstset{language=C++,frameround=fttt}
    \begin{lstlisting}[frame=trBL]
        typedef struct {
        string mailDir = "";
        string usersFile = "";
        bool isHashed = true;
        string clientUser = "";
        string serverUser = "";
        string clientPass = "";
        string serverPass = "";
        int commSocket = -1;
        string pidTimeStamp = "";
        bool authorized = false;
        } threadStruct;
    \end{lstlisting}
    Struktura obsahuje:
    \begin{description}
        \item [string mailDir] kořenovou složku e-mailového adresáře
        \item [string usersFile] cestu k souboru s autorizačními údaji klienta
        \item [bool isHashed] formu autorizace
        \item [string clientUser] jméno klienta, které obdrží po komunikaci s klientem
        \item [string serverUser] jméno klienta, které bylo uloženo v souboru s autorizačními údaji, je zde pro urychlení práce s danými daty
        \item [clientPass] heslo klienta, které obdrží po komunikaci s klientem
        \item [serverPass] heslo klienta, které bylo uloženo v souboru s autorizačními údaji, je zde pro urychlení práce s danými daty
        \item [int commSocket] komunikační socket, pro komunikaci s daným klientem, při inicializaci má hodnotu -1
        \item [string pidTimeStamp] speciální nonce, která je po startu komunikace klienta se serverem zaslána klientovi
        \item [bool authorized] stav autorizace klienta
    \end{description}
    \label{subsec:mailstruct}
    Druhou strukturou, která je v serveru použita, je struktura e-mailu, její definice je následovná:
    \lstset{language=C++,frameround=fttt}
    \begin{lstlisting}[frame=trBL]
        struct mailStruct{
        unsigned long id;
        string name;
        size_t size;
        string dir;
        bool toDelete;
        } *mailStructPtr;
    \end{lstlisting}
    Struktura obsahuje následující položky:
    \begin{description}
        \item [unsigned int id] id jednotlivých e-mailů
        \item [string name] jméno e-mailu
        \item [size\_t size] velikost daného e-mailu v oktetech \ref{subsec:mailsize}
        \item [string dir] cesta ke kořenové složce e-mailového adresáře
        \item [bool toDelete] příznak označující e-mail ke smazání
    \end{description}
    Jednotlivá vlákna jsou pro usnadnění práce s nimi uložena v jednom vektoru vláken s názvem threads. Vektorem můžeme označit dynamické pole umožňující rychlý a efektivní přístup k jednotlivým prvkům.\par
    Jednotlivé e-maily jsou pro usnadnění práce s nimi uloženy v jednom globálním listu nazvaném mailList. List je provázaný seznam prvků umožňující nám snadnou iteraci skrze jednotlivé prvky.
    \subsubsection{Implementace programu}
    \label{subsec:serverimplementation}
    Server je implementován jako TCP server, který po spuštění provede kontrolu argumentů a jejich rozparsování za pomocí funkcí: \texttt{checkParams()} a \texttt{parseParams()}. Pokud nejsou správně zadány parametry odpovídající jednomu ze tří běhů aplikace (\ref{subsec:makeandrun}), dojde k ukončení programu s chybovým kódem \textbf{EXIT\_SUCCESS} (\ref{itm:exitsuccess}), V opačném případě se pokračuje v běhu programu. Dojde ke kontrole parametrů \textsc{help} (\ref{itm:help}) a \textsc{reset} (\ref{itm:reset}). V dalším kroku se provede validace souboru s autorizačními údaji uživatele serveru (\ref{subsec:authconfig}). Pokud soubor není v pořádku, server vrací návratový kód \textbf{EXIT\_FAILURE} (\ref{itm:exitfailure}). Po úspěšné kontrole všech těchto parametrů se začne inicializovat server, nejprve se vytvoří \textsc{serverSocket} za pomocí funkce \texttt{socket()}, následně je tento socket nastaven funkcí \texttt{setsockopt()}. Pokud je vše v pořádku, server vytvoří \textsc{clientaddr} a \textsc{serveraddr} proměnné typu \textit{\textsc{sockaddr\_in}}, které se naplní vhodnými hodnotami, jako typ spojení a číslo portu. Poté je za pomocí funcke \texttt{bind()} socket nabindován a začne poslouchat na daném socketu s názvem \textsc{serverSocket}.
    Pro chování TCP serveru je využit nekonečný \textsc{while} cyklus na jehož počátku dojde k přijetí spojení funkcí \texttt{accept()}, následně dojde k vytvoření a naplnění struktury \textit{\textsc{threadStruct}}, následně dojde k vytvoření vlákna a přístupu do funkce \texttt{clientThread()}. Pokud došlo k chybě při přijetí spojení, cyklus se ukončí a program končí. Pokud ne, přechází se už k funkčnosti samotného POP3 serveru, která je podrobně řešena v sekci \ref{subsec:pop3implementation}
    \subsubsection{Implementace logiky POP3}
    \label{subsec:pop3implementation}
    Ve funkci \texttt{clientThread()} se provádí příjem jednotlivých zpráv, je zde tedy druhý nekonečný cyklus (tentokráte \textsc{for}), který doplňuje logiku serverového chování. Při přijetí spojení se klientovi okamžitě zašle nonce (\ref{subsec:nonce}), na základě které klient ví, že je připojen a odpovídá na ni e-mailovými operacemi. Nyní je server v autorizačním stavu (\ref{subsec:serverstatus}). V tomto momentě může klient zasílat pouze APOP, USER, PASS nebo QUIT příkazy. Funkce je vysvětlena v sekci \ref{subsec:mailoperations}. Dokud se klient neautorizuje korektními údaji, nemůže přejít do dalšího, transakčního (\ref{subsec:serverstatus}), stavu. Po úspěšné autorizaci klienta dojde k výhradnému uzamčení e-mailového adresáře funkcí \texttt{lockMaildir()}, může jej tedy využít pouze daný klient, nikdo jiný. Pokud se autorizace nezdaří, neobdrží přístup a musí se autorizovat znovu. Celou autorizaci řeší funkce \texttt{authorizeUser()}, která vrací hodnotu \textsc{BOOL}, ta je uložena do struktury daného vlákna (\ref{subsec:threadstruct}) pro každého klienta zvlášť. \par
    Po uzamčení e-mailového adresáře dojde k vytvoření listu e-mailů pomocí funkce \texttt{createListFromMails()} z e-mailů v adresáři /new a pokud se nejedná o první spuštění serveru, musí být přítomen i konfigurační soubor e-mailu obsahující starší e-maily, které byly ve složce /cur, které si také načte funkcí \texttt{loadMailsFromCfg()}. Během tohoto tvoření listu jsou e-maily čteny, je zjišťována jejich velikost funkcí \texttt{getFileSize()}(?? - reference na size e-mailů) a zároveň, pokud se jedná o e-maily ve složce /new, přesouvány do složky /cur. Po dokončení této operace přestoupí server do transakčního stavu přes funkci \texttt{executeMailServer()}, která řeší logiku celého POP3 serveru.\par
    Tato funkce obdrží vždy operaci, která má být vykonávána a zprávu, která byla od klienta přijata, tu zpracuje a na základě ní odešle adekvátní reakci. V této fázi může přijímat pouze příkazy: LIST, NOOP, STAT, RETR, DELE, RSET, UIDL, TOP a QUIT (\ref{subsec:mailoperations}). Každá z těchto operací má svou vlastní C++ funkci, přesněji: \texttt{listOperation()}, \texttt{noopOperation()}, \texttt{statOperation()}, \texttt{retrOperation()}, \texttt{deleOperation()}, \texttt{rsetOperation()}, \texttt{uidlOperation()}, \texttt{topIndexOperation()}, \texttt{quitOperation()}, výjimkou jsou operace LIST, UIDL. Tyto příkazy mají dvě funkce, protože jsou rozlišeny podle počtu argumentů. Jedná se tedy o funkce: \texttt{listIndexOperation()} a \texttt{uidlIndexOperation()}. Implementace těchto funkcí je řešena dle specifikace RFC (?? - reference na RFC POP3), které je věnována sekce \ref{subsec:mailoperation}.\par
    Klient komunikuje s klientem tak dlouho, dokud klient nezašle příkaz QUIT, načež server přejde do stavu aktualizace, kdy dojde ke smazání všech e-mailů, které byly označeny ke smazání, v e-mailovém adresáři. Tento stav je prováděn funkcí \texttt{quitOperation()}, která provede smazání všech e-mailů funkcí \texttt{deleteMarkedForDeletion()}. Tato funkce zkontroluje všechny e-maily v listu a ty které byly označeny příznakem toDelete (\ref{subsec:mailstruct}), smaže.\par
    Pokud smazání proběhlo v pořádku, server pošle klientovi zprávu obsahující počet e-mailů, které v e-mailovém adresáři zůstali a odpojí jej. Pokud došlo k problému při mazání, server na tento fakt upozorní klienta a taktéž jej odpojí.\par
    Pokud klient nebyl před zasláním příkazu QUIT v transakčním stavu, musel být ve stavu autorizačním, poté dojde k použití stejné funkce, jelikož ale nebyl list naplněn, žádný e-mail se nesmaže a je klient odpojen.
    Veškerá vlákna klientů se mažou a uvolňuji z paměti až při ukončení serveru (\ref{subsec:sigint}).
    \subsubsection{Implementace ukončení běhu serveru}
    \label{subsec:sigint}
    Ukončení běhu serveru je možné pouze v případě, že je serveru předán signál \textbf{SIGINT}. Takový signál je signál označený jako vyvolaný z klávesnice. Je to ukončující signál, jeho účelem je terminovat proces, potažmo server. Lze jej vyvolat z klávesnice stisknutím kláves: CTRL\^{}C. Při takovémto ukončení dojde k odpojení veškerých klientů přes funkci \texttt{closeThreads()}, korektně se uloží veškeré informace o e-mailech do konfiguračního souboru (\ref{subsec:config}) s e-maily užitím funkce \texttt{createMailCfg()}, uvolní se veškerá struktury z paměti za užití funkce \texttt{disposeList()} a server se ukončí s návratovým kódem \textbf{EXIT\_SUCCESS}. (\ref{itm:exitsuccess}). Toto vše je nutné pro korektní ukončení programu a následné bezproblémové spuštění při opakovaném spuštění serveru.
    \clearpage
    \section{Neimplementované součásti}
    \label{sec:nonimplementedsections}
    Nebyla implementována součást automatického odhlášení ze sekce 3, RFC 1939 \cite{POP3}.
    \section{Implementovaná rozšíření}
    \label{sec:implementedsections}
    Proběhla implementace příkazu TOP dle sekce 7, RFC 1939 \cite{POP3}.
    \bibliographystyle{plain}
    \bibliography{references}
\end{document}
